
\chapter{Controlling sequence timing}
\label{ch:timing}

The default core duration is set to the value specified in \texttt{cores.txt}, however the duration can be extended in real-time by setting the value of the 'textra' column in {\tt scanloop.txt} to a non-zero value.
This allows the sequence timing to be controlled dynamically, e.g., for the purpose of varying TE or TR during a scan.

If the minimum core duration exceeds the prescribed duration in {\tt cores.txt}, the minimum core duration is used (without warning).
It is therefore perfectly fine to set the core duration in {\tt cores.txt} to '0', since this guarantees that the minimum duration will be used which is often the desired behavior.

Figure~\ref{fig:timing} shows detailed timing information for the three core types: gradients-only, RF excitation, and data acquisition.
For gradients-only cores, the minimum core duration is equal to the waveform duration plus the controls variables (CVs) 'start\_core', 'timetrwait', and 'timessi'.
These have been determined empirically, and are currently set to 224us, 64us, and 100us, respectively.
For RF and acquisition cores, the core duration must be extended by 'myrfdel' and 'daqdel', respectively, to account for gradient delays with respect to RF transmission and data acquisition, respectively.
%
\myfigure{timing}{0.75}{
Detailed timing diagram for the three core types:
(a) gradients-only, (b) RF excitation, and (c) data acquisition.
The labels correspond to Control Variables (CVs) in{\toppe}.
Note that \texttt{scansim.m} uses timing values in \texttt{timing.txt} (e.g., \texttt{example/timing.txt}) to reproduce the precise sequence timing one should expect to observe on the scanner.
}
%



